\chapter{Executive Summary (Jacob)}
This is a report on the process of analyzing and implementing a Solar Panel system, in close dialogue with the costumer, to get his exact needs for the system-to-be. The customer wants this product as a showcase for high-school visitors, so they can see how to produce green energy.
This report will discuss power line communication, LPC2478 programming and many other exciting topics, in the process of producing this solar panel system, which is only a small part of a bigger, more complex system. Other Teams creates different system parts (modules), which at the end have to work together and interact with each other.
\chapter{Preface (Jacob)}
This report is aimed at lectors and sensors at the Electronic Design Engineering study program at AU- Herning, to share the process of creating the Solar Panel system.
The report was prepared at 3rd and 4th semester of the EDE study, from 08-11 to 05-12. Thank you to classmates and cooperative teams at E10 class for sharing knowledge and experiences when problems occurred, and also for being structured and conscientious when developing the common parts of the project.
The lectors at the AU – Herning has been helpful and willing to help and sharing their knowledge, it is their job to do so, but a big thanks to them for doing a great job teaching has to be given!
\chapter{Version History}

\textbf{0.1}\\
All steps in the launch phase were followed in order to gain knowledge regarding the system that is going to be built.

\begin{itemize}
\item Launch Phase
\subitem The first approach of the report is made.
\end{itemize}

\textbf{0.2}\\

Meeting with the teachers was made, where the discussed topics were: changes and modifications.
\begin{itemize}
\item History added.
\item Summary added
\item Preface added
\item Introduction added
\item Blocks/events updated throughout the entire report.
\end{itemize}

\textbf{0.3}\\

Even more discussion with the teachers gave more additions to the report.
\begin{itemize}
\item Words like “we” and “our” were translated into science language. (3rd person passive)
\item Block Event table, blocks/events switched
\item State Machine Diagram Updated
\item Common Requirement document made.
\item Requirements updated.
\item Design Criteria, safety added.
\item Technical platform updated. Emergency added!
\item Product Acceptance added
\end{itemize}

\textbf{0.4}\\
And a final walkthrough of the Launch Phase report.
\begin{itemize}
\item Grammar and Language revised
\item User needs/requirements updated and corrected
\item Updated Blocks/events
\item Block Diagram Added
\item State machine diagrams added (for all blocks)
\item Update the system interface analysis
\item Updated function analysis
\item Improved requirement analysis
\end{itemize}

\textbf{0.5 (15/12 2011)}\\

Collect pre project, launch phase and realization phase in a single document, updating the intro to the report, to fit all phases.
\begin{itemize}
\item Phases collected in a single document.
\item Structure and order fixed.
\item Introduction updated.
\item Executive Summary Updated.
\item Preface Updated.
\item Problem Statement Updated
\item Introduction to the different phases added.
\end{itemize}

\chapter{Introduction}

\begin{itemize}
\item was the project initiated?
\item ideas, interests and thoughts are behind your choice of subject?
\item others worked on the problem and what did they do?
\item introduction may include artifacts from the PreProject, e.g. Rich Picture. It may fill two pages and cover the problem widely
\item may be a good idea to write this part at the end of a project period
\item Remember to include a list of abbreviations, e.g. in a table under  ”Introduction” or alternatively under ”Appendices”
\item describe how references can be found in the report (i.e. in square brackets).
\end{itemize}

\tableofcontents